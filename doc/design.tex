\documentclass{mydoc}

\author{张丘洋}
\title{ComboTree 实现}

\begin{document}

\maketitle

\section{概览}

根据 ComboTree 的设计,将数据结构分为 \code{CLevel}、\code{BLevel} 和 \code{ALevel}。
其中 \code{BLevel::Entry} 中的键值对缓存与 \code{CLevel::Node} 使用相同的数据结构
\code{KVBuffer} 来保存数据。

\section{KVBuffer}

\code{KVBuffer} 使用\textbf{前缀压缩}保存若干键值对数据。其数据结构定义为:

\begin{codes}{C++}
struct KVBuffer {
  union {
    uint16_t meta;
    struct {
      uint16_t prefix_bytes : 4;  // 共同前缀字节数
      uint16_t suffix_bytes : 4;  // 后缀字节数,与前缀之和为8
      uint16_t entries      : 4;  // 目前保存的键值对数
      uint16_t max_entries  : 4;  // 最大可保存的键值对数
    };
  };
  uint8_t buf[112];
};
\end{codes}

\code{KVBuffer} 包含了 2 个字节的元数据 \code{meta},以及 112 个字节的 \code{buf}
来保存键值对。键值对的保存是\textbf{排序}的。

当后缀字节数为 1,一个 \code{KVBuffer} 能够保存 12 个键值对;当后缀字节数为 2
时,能够保存 11 个键值对。

由于使用了前缀压缩,键的大小是不固定的,但是值的大小固定为 8 个字节,为了使值
在保存时能够 8 字节\textbf{对齐},\code{buf} 从左到右顺序保存键,从右到左顺序保存值,如
下图所示。

\begin{figure}[!htbp]
\centering
\begin{tikzpicture}[x=1em,y=1em,every node/.style={draw,line width=1pt,inner sep=.5em,minimum height=2.2em,minimum width=2.5em}]
  \node at (0,0)                          (0) {key(0)};
  \node[right=-1pt of 0.east,anchor=west] (1) {key(1)};
  \node[right=-1pt of 1.east,anchor=west] (2) {...};
  \node[right=-1pt of 2.east,anchor=west] (3) {key(n)};
  \node[right=-1pt of 3.east,anchor=west] (4) {...};
  \node[right=-1pt of 4.east,anchor=west] (5) {value(n)};
  \node[right=-1pt of 5.east,anchor=west] (6) {...};
  \node[right=-1pt of 6.east,anchor=west] (7) {value(1)};
  \node[right=-1pt of 7.east,anchor=west] (8) {value(0)};
\end{tikzpicture}
\end{figure}

\section{CLevel}

\begin{codes}{C++}
// sizeof(Node) == 128
struct __attribute__((aligned(64))) Node {
  enum class Type : uint8_t {
    INVALID,
    INDEX,
    LEAF,
  };

  Type type;
  uint8_t __padding[7]; // used to align data
  union {
    // uint48_t pointer
    uint8_t next[6];    // used when type == LEAF. LSB == 1 means NULL
    uint8_t first_child[6]; // used when type == INDEX
  };
  union {
    // contains 2 bytes meta
    KVBuffer leaf_buf;  // used when type == LEAF, value size is 8
    KVBuffer index_buf; // used when type == INDEX, value size is 6
  };
};

// sizeof(CLevel) == 6
class CLevel {
  ......
 private:
  uint8_t root_[6]; // uint48_t pointer, LSB == 1 means NULL
};
\end{codes}

CLevel 采用的是 \textbf{B+ 树},每个节点大小为 \textbf{128 字节,两个 Cache 行},
与 \code{BLevel::Entry} 的大小相同。节点使用 \code{KVBuffer} 来保存键值对和子节点。
节点\textbf{没有保存父指针},通过函数的递归来传递父指针,这样避免了扩展时父指针的修改,
也减少了父指针的存储。

C 层的中间节点在保存子节点\textbf{指针时使用 6 个字节来保存},也就是 \code{KVBuffer}
键值对中的值大小为6,从而有更大的扇出。

无论是子节点指针还是根节点指针都通过将指针的\textbf{最低比特位置 1} 来表示 NULL,
这样避免了多字节的拷贝,并且 \emph{Optimizing Systems for Byte-Addressable NVM
by Reducing Bit Flipping} 论文中提到这种方式可以减少比特翻转从而提高 NVM 寿命。

整个 C 层 \code{KVBuffer} 的\textbf{前缀压缩字节数是一样的},前缀字节数和共同前缀可以由
对应的 \code{BLevel::Entry} 来确定。

\textbf{C 层无锁},由 \code{BLevel::Entry} 对应的锁来提供并发访问控制。

\section{BLevel}

B 层的主要数据结构定义如下所示:

\begin{codes}{C++}
// sizeof(Entry) == 128
struct __attribute__((aligned(64))) Entry {
  uint64_t entry_key;     // 8 bytes
  CLevel clevel;          // 6 bytes
  KVBuffer<48+64,8> buf;  // 114 bytes
};

struct BRange {
  uint64_t start_key;
  uint32_t logical_entry_start;
  uint32_t physical_entry_start;
  uint32_t entries;
};

class BLevel {
  ......
 private:
  ......
  Entry* __attribute__((aligned(64))) entries_;
  size_t nr_entries_;
  std::atomic<size_t> size_;

  // 锁粒度为 Entry
  std::shared_mutex* lock_;

  // B 层分为 EXPAND_THREADS 个 range,这里多分配一个 range 减少编程难度
  BRange ranges_[EXPAND_THREADS+1];

  // 每个 range 分为多个小区间,记录每个小区间中的键值对数
  uint64_t interval_size_;
  uint64_t intervals_[EXPAND_THREADS];
  std::atomic<size_t>* size_per_interval_[EXPAND_THREADS];

  // 记录扩展时每个区间已经扩展的 entry 数目以及最大的键
  static std::atomic<uint64_t> expanded_max_key_[EXPAND_THREADS];
  static std::atomic<uint64_t> expanded_entries_[EXPAND_THREADS];
};
\end{codes}

\section{ALevel}

\section{NVM 管理}

\end{document}